%设置页面边距(word标准页面)
\documentclass[a4paper]{article}
\usepackage{geometry}
\geometry{a4paper,left=2.7cm,right=2.7cm,top=2.54cm,bottom=2.54cm}

%导入ctex包
\usepackage[UTF8,heading=true]{ctex}

%设置摘要格式
\usepackage{abstract}
\setlength{\abstitleskip}{0em}
\setlength{\absleftindent}{0pt}
\setlength{\absrightindent}{0pt}
\setlength{\absparsep}{0em}
\renewcommand{\abstractname}{\textbf{\zihao{4}{摘要}}}
\renewcommand{\abstracttextfont}{\zihao{-4}} %设置摘要正文字号

%设置页眉和页脚,只显示页脚居中页码
\usepackage{fancyhdr}
\pagestyle{plain}

%调用数学公式包
\usepackage{amssymb}
\usepackage{amsmath}

%调用浮动包
\usepackage{float}
\usepackage{subfig}
\captionsetup[figure]{labelsep=space} %去除图标题的冒号
\captionsetup[table]{labelsep=space} %去除表格标题的冒号
%设置标题格式
\ctexset {
	%设置一级标题的格式
	section = {
		name={,、},
		number=\chinese{section}, %设置中文版的标题
		aftername=,
	},
	%设置三级标题的格式
	subsubsection = {
		format += \zihao{-4} % 设置三级标题的字号
	}
}


%使得英文字体都为Time NewTown
%\usepackage{times}

%图片包导入
\usepackage{graphicx}
\graphicspath{{figures/}} %图片在当前目录下的figures目录

%参考文献包引入
\usepackage{cite}
\usepackage[numbers,sort&compress]{natbib}

%代码格式
\usepackage{listings}
\usepackage{graphicx}%写入python代码
\usepackage{pythonhighlight}%python代码高亮显示
\lstset{
	%numbers=left, %设置行号位置
	%	numberstyle=\tiny, %设置行号大小
	keywordstyle=\color{blue}, %设置关键字颜色
	commentstyle=\color[cmyk]{1,0,1,0}, %设置注释颜色
	escapeinside=``, %逃逸字符(1左面的键),用于显示中文
	breaklines, %自动折行
	extendedchars=false, %解决代码跨页时,章节标题,页眉等汉字不显示的问题
	xleftmargin=1em,xrightmargin=1em, aboveskip=1em, %设置边距
	tabsize=4, %设置tab空格数
	showspaces=false %不显示空格
}


\renewcommand{\refname}{}

%item包
\usepackage{enumitem}

%表格加粗
\usepackage{booktabs}

%设置表格间距
\usepackage{caption}

%允许表格长跨页
\usepackage{longtable}

%伪代码用到的宏包
\usepackage{algorithmic}
\usepackage{algorithm}

%正文区
\title{“板凳龙” 闹元宵} 
\date{} %不显示日期

%文档
\begin{document}
	\maketitle
	\vspace{-6em} %设置摘要与标题的间距
	\zihao{-4} %设置正文字号
	%摘要部分
	\begin{abstract}
		在“板凳龙”的舞龙过程中,舞龙队应当保持尽可能小的间距和较快的行进速度,以达到较好的观赏性。在题设要求下,通过对“板凳龙”问题的进行分析与适当建模,我们在等距螺线下求出了板凳龙的最小螺距和最大速度。
		
		\hspace{0.2em}\textbf{针对问题一},我们建立了合适的\textbf{极坐标系}用以刻画舞龙队的行进位置和速度。首先,基于已知的把手位置,通过不断对前一把手进行\textbf{二分查找},我们迭代查找确定了舞龙队所有把手的初始物理位置。其次,通过建立等距螺线和速度间关系的\textbf{微分方程},求解出了\textbf{舞龙队龙头的位置和行进时间的关系},\textbf{再次运用二分查找}即可得到各把手在各个时间下的位置。接着,通过对舞龙队的各把手的位置使用\textbf{微元法},我们得到了舞龙队在各个位置下的速度。最后,我们利用python将舞龙队的行进过程可视化表现出来,对我们的建模进行了验证,并将计算相关数值填入了表1和文件中。
		
		\textbf{针对问题二},我们
		
		\textbf{针对问题三},...
		
		\textbf{针对问题四},...
		
		\textbf{针对问题五},...\\
		%关键词(上文最后一段要用“\\”换行)
		\newline
		\noindent{\textbf{关键词:} \textbf{关键词1}\quad   \textbf{关键词2}\quad \textbf{关键词3} \quad \textbf{关键词4}\quad \textbf{关键词5}} 
	\end{abstract}
	
	\clearpage %换页
	
	%正文部分
	%Part one
	\section{问题背景与重述}
	\subsection{问题背景}
	“板凳龙”是浙闽地区的传统民俗活动,其舞龙队能够将上百条板凳螺旋式盘绕形成巨龙。舞龙队的盘绕路线大致呈螺线状向内部盘绕。其中,螺距的最小化与速度的最大化不仅影响舞龙队形的紧凑度和动态美,也直接关系到表演的安全性。通过对这些参数的深入研究,可以优化舞龙表演的编排,确保参与者的安全,为传统文化的传承与发展提供科学支撑。
	
	%\begin{figure}[H]
	%	\centering %图片居中
	%	\captionsetup{skip=4pt} % 设置标题与表格的间距为4pt
	%	\includegraphics[width=10cm]{图片文件名} %width设置图片大小
	%	\caption{商超蔬菜示意图\label{商超蔬菜示意图}} %设置图片的标题及引用标签
	%\end{figure}
	
	\subsection{问题重述}
		\textbf{问题一:}在题目给定的 55cm 螺距的等距螺线作为行进路线的情形下,取龙头位置为第十六圈与 x 轴交点处为龙头的初始位置,且龙头以 1m/s 的速度向内顺时针盘绕,计算 0-300s 内每一秒舞龙队各把手的所在位置以及速度。
		
		\textbf{问题二:}
		
		\textbf{问题三:}
		
		\textbf{问题四:}
		
		\textbf{问题五:}
	
	%Part Two
	\section{问题分析}
	\subsection{问题一的分析}
		问题一要求我们在给定情景下分析从初始时刻到 300s 为止的时间中舞龙队各把手的中心位置和速度。首先,我们推导出了描述等距螺线路径与速度关系的微分方程,并解出了龙头位置与行进时间的函数关系。接着,利用已知的把手位置,我们采用了二分查找算法,迭代地确定了舞龙队所有把手的具体物理位置,从而计算出每一把手在不同时间点的位置。然后,我们应用微元法分析了舞龙队各把手的位置变化,从而得出了它们的速度。最终,我们使用Python编程语言对舞龙队的运动过程进行了可视化模拟,以验证模型的正确性。
	\subsection{问题二的分析}
	\subsection{问题三的分析}
	\subsection{问题四的分析}
	\subsection{问题五的分析}
	
	%Part Three
	\section{模型假设}
	%假设的列表
	\begin{enumerate} 
		\item 假设板凳龙盘入与盘出时,龙头、各节龙身以及龙尾之间连接牢固,旋转润滑,运动连续不停顿
		\item 假设不考虑板凳厚度对板凳龙整体运动产生的影响,
		\item 假设板凳龙的各板凳均位于同一平面上
	\end{enumerate}
	
	%Part Four
	\section{符号说明}
	%浮动体表格,使用table实现
	\begin{table}[H] %[h]表示在此处添加浮动体,默认为tbf,即页面顶部、底部和空白处添加
		\captionsetup{skip=4pt} % 设置标题与表格的间距为4pt
		\centering
		\setlength{\arrayrulewidth}{2pt} % 设置表格线条宽度为1pt
		\begin{tabular}{cc} %c表示居中,l表示左对齐,r表示右对齐,中间添加“|”表示竖线
			\hline
			\makebox[0.15\textwidth][c]{符号} & \makebox[0.6\textwidth][c]{说明}  \\ 
			\hline
			$\rho$ & 极坐标下点的模长  \\
			$\theta$ & 极坐标下点的辐角  \\
			$D$ & 等距螺线螺距  \\
			$L_1$ & 龙头板凳长  \\
			$L_2$ & 龙身龙尾板凳长  \\
			$l$ & 板凳板头长 \\
			$d$ & 板凳宽  \\
			$t$ & 舞龙队运动时间  \\
			$A_0$ & 龙头把手的中心点  \\
			$A_i$ & 第 i 节龙身靠前把手的中心点  \\
			$A_{223}$ & 龙尾后侧把手中心点  \\
			\hline
		\end{tabular}
		% \hline是横线,采用\makebox设置列宽
	\end{table}
	
	
	%Part Five
	\section{模型的建立与求解}
	\subsection{问题一模型的建立与求解}
	\subsubsection{由龙头把手确定其余把手的位置}
	
		依据题意,舞龙队的行进路线由 $D = 55cm$ 的等距螺线确定。选用极坐标刻画行进路线时,该路线可以由以下方程确定:
		
		\begin{equation}
			\rho = \frac{D}{2\pi}\theta, \quad \theta \in [0,+\infty)
		\end{equation}
		
		使用形如 $(\rho, \theta)$ 的方式表示点的坐标,因此初始点 A 点的坐标可以表示为 $( 16 \cdot 0.55, 16 \cdot 2\pi )$。
		
		题设给定的初始条件中,只有龙头把手的位置是已知信息,因此需要得到龙头把手的位置和其余所有把手的位置关系才能确定其余把手的位置。通过解析可以发现:该关系难以求出解析解。因此可以考虑\textbf{从龙头把手位置逐级递推到每一把手的位置}。
		
		为了做到这一点,我们选用了二分查找算法,在任一把手 $A_i$ 位置已知的情况下,查找下一把手所在位置。即给定 $\theta_{i}$ ,确定 $\theta_{i+1}$ 的大小。而 $\theta_{i+1}$ 合适当且仅当两把手的间距正好等于板凳两孔中心的距离。根据极坐标下的余弦定理,两点间的位置即为:
		
		$$\Delta l = \sqrt{\rho_i^2 + \rho_{i+1}^2 - 2\rho_i \rho_{i+1} cos(\theta_i - \theta_{i+1})}$$
		
		代入 $\rho$ 并令 $\Delta l$ 分别减去龙头和龙身除去板头的长度,即分别减去 $L_1 - 2 \cdot l$ 和 $L_2 - 2 \cdot l$,即可得到二分查找的目标函数 $f$ ,其零点即为 $\theta_{i+1}$ 的二分查找值:
		
		\begin{equation}
			f(\theta) = \frac{D}{2\pi} \sqrt{\theta^2 + \theta_i^2 - 2\theta \theta_i cos(\theta - \theta_i)} - (L_j - 2l), \quad j=1, 2
		\end{equation} 
		
		目标函数 $f$ 二分查找所得的零点即为 $\theta_{i+1}$ 的数值解。
		
		二分求解代码见附录2 % TODO!!!!!!!!!!!!!!!!!!!!!!!!!!!!!!!!!!!!!!!!!!!!!!!!!!!!!!!!!!11↑
	
	\subsubsection{确定龙头把手的位置与时间关系}
	
		5.1.1 小节提供了由龙头把手计算其余各个把手位置的方法,因此只需要确定每一秒龙头把手的位置,利用上一小节的方法,即可得到每一秒钟各个把手的位置信息。
		
		极坐标下,龙头把手位置可以由 $\theta$ 参数唯一确定。因此,只需要求解出龙头的辐角 $\theta$ 与时间 $t$ 的关系即可。对等距螺线使用微元法,可以用 $d\theta$ 和 $d\rho$ 表示出螺线弧微元 $ds$:
	
		$$ ds = \sqrt{(d\rho)^2 + (\rho d\theta) ^ 2}$$
		
		而微元 $ds$ 等于龙头在 $dt$ 时间内行进的距离,即:
		
		\begin{equation}
			\sqrt{(d\rho)^2 + (\rho d\theta) ^ 2} = vdt
		\end{equation}
		
		联立 (1) 和 (3) 式消去 $\rho$ 即可解得龙头的辐角 $\theta$ 和时间 $t$ 满足以下关系:
		
		\begin{equation}
			\frac{D}{2\pi}(-\frac{1}{2}ln(\sqrt{\theta^2+1}-\theta)+\frac{1}{2}\theta\sqrt{\theta^2+1})=vt+C
		\end{equation}
		
		% TODO 常数 C 需要进一步计算
		
		利用数值计算即可求得给定时刻下龙头把手的位置。
	
	\subsubsection{确定各把手的速度}
		
		5.1.2 小节计算出了各个把手的位置在给定时刻下的数值解,而速度是位置在时间尺度下的微元。因此,只需要取时间微元 $\Delta t$ ,并计算在给定时间 $t_1$ 两侧的 $\Delta t$ 下的把手位置的位移模长 $|\vec{x}(t_1+\Delta t) - \vec{x}(t_1 - \Delta t)|$ ,将其除以两倍的 $\Delta t$ 即可得到速度大小。
		
		
		
		具体来说,对于给定的时间 $t_1$ ,把手 $A_i$ 此时的速度大小的数值解为:
		
		\begin{equation}
			v_{t1} = \frac{\sqrt{(x_i(t_1 + \Delta t) - x_i(t_1 - \Delta t))^2 + (y_i(t_1 + \Delta t) - y_i(t_1 - \Delta t))^2}}{2\Delta t}
		\end{equation}
		
	% TODO: \subsubsection{模型的检验}
		
		利用这种方式计算出问题一中的各个把手的位置和速度见 result1.xlsx 文件。其中 0 s、60 s、120 s、180 s、240 s、300 s 时,龙头前把手、龙头后面第 1、51、101、151、201 节龙身前把手和龙尾后把手的位置和速度如下表所示:
		
		
		\begin{table}[H] %[h]表示在此处添加浮动体,默认为tbf,即页面顶部、底部和空白处添加
			\captionsetup{skip=4pt} % 设置标题与表格的间距为4pt
			\caption{问题一的位置结果}
			\centering
			\setlength{\arrayrulewidth}{0.5pt} % 设置表格线条宽度为1pt
			\begin{tabular}{|c|c|c|c|c|c|c|} %c表示居中,l表示左对齐,r表示右对齐,中间添加“|”表示竖线
				\hline
				% \makebox[0.15\textwidth][c]{符号} & \makebox[0.6\textwidth][c]{说明}  \\ 
				% \hline
				& 0 s & 60 s & 120 s & 180 s & 240 s & 300 s \\ \hline
				龙头 x (m)         & 8.800000 &	5.799209 &	-4.084887 &	-2.963609 &	-0.818702 &	4.420274 \\ \hline
				龙头 y (m)         & -0.000000 &	-5.771092 &	-6.304479 &	6.094780 &	5.590600 &	2.320429 \\ \hline
				第 1 节龙身 x (m)  & 8.363824 &	7.456758 &	-1.445473 &	-5.237118 &	-3.469210 &	2.459489 \\ \hline
				第 1 节龙身 y (m)  & 2.826544 &	-3.440399 &	-7.405883 &	4.359627 &	4.516167 &	4.402476 \\ \hline
				第 51 节龙身 x (m) & -9.518732 &	-8.686317 &	-5.543150 &	2.890455 &	-6.560125 &	-6.301346 \\ \hline
				第 51 节龙身 y (m) & 1.341137 &	2.540108 &	6.377946 &	7.249289 &	1.969759 &	0.465829 \\ \hline
				第 101 节龙身 x (m) &2.913983 &	5.687116 &	5.361939 &	1.898794 &	0.218823 &	-6.237722 \\ \hline
				第 101 节龙身 y (m) &-9.918311 &	-8.001384 &	-7.557638 &	-8.471614 &	7.831999 &	3.936008 \\ \hline
				第 151 节龙身 x (m) &10.861726 &	6.682311 &	2.388757 &	1.005154 &	4.451294 &	7.040740 \\ \hline
				第 151 节龙身 y (m) &1.828753 &	8.134544 &	9.727411 &	9.424751 &	-7.486030 &	4.393013 \\ \hline
				第 201 节龙身 x (m) &4.555102 &	-6.619664 &	-10.627211 &	-9.287720 &	-1.731014 &	-7.458662 \\ \hlin
				第 201 节龙身 y (m) &10.725118 &	9.025570 &	1.359847 &	-4.246673 &	9.344557 &	-5.263384 \\ \hline
				龙尾(后) x (m)    &-5.305444 &	7.364557 &	10.974348 &	7.383896 &	7.057739 &	1.785033 \\ \hline
				龙尾(后) y (m)    &-10.676584 &	-8.797992 &	0.843473 &	7.492370 &	-6.846021 &	9.301164 \\ \hline
			\end{tabular}
			% \hline是横线,采用\makebox设置列宽
		\end{table}
		
		\begin{table}[H] %[h]表示在此处添加浮动体,默认为tbf,即页面顶部、底部和空白处添加
		\captionsetup{skip=4pt} % 设置标题与表格的间距为4pt
		\caption{问题一的速度结果}
		\centering
		\setlength{\arrayrulewidth}{0.5pt} % 设置表格线条宽度为1pt
		\begin{tabular}{|c|c|c|c|c|c|c|} %c表示居中,l表示左对齐,r表示右对齐,中间添加“|”表示竖线
			\hline
			% \makebox[0.15\textwidth][c]{符号} & \makebox[0.6\textwidth][c]{说明}  \\ 
			% \hline
			& 0 s & 60 s & 120 s & 180 s & 240 s & 300 s \\ \hline
			龙头 (m/s)      &    1.000000 &	1.000000 &	1.000000 &	1.000000 &	1.000000 &	1.000000 \\ \hline
			第 1 节龙身 (m/s) &  0.999971 &	0.999961 &	0.999945 &	0.999917 &	0.999859 &	0.999709 \\ \hline
			第 51 节龙身 (m/s) & 0.999742 &	0.999662 &	0.999540 &	0.999331 &	0.998940 &	0.998064 \\ \hline
			第 101 节龙身 (m/s) &0.999575 &	0.999455 &	0.999277 &	0.998971 &	0.998436 &	0.997302 \\ \hline
			第 151 节龙身 (m/s) &0.999451 &	0.999303 &	0.999082 &	0.998726 &	0.998121 &	0.996860 \\ \hline
			第 201 节龙身 (m/s) &0.999352 &	0.999190 &	0.998942 &	0.998552 &	0.997902 &	0.996575 \\ \hline
			龙尾(后) (m/s) &   0.999317 &	0.999143 &	0.998889 &	0.998490 &	0.997827 &	0.996476 \\ \hline
		\end{tabular}
		% \hline是横线,采用\makebox设置列宽
		\end{table}
	
	
	
	
	
		
	
		计算公式模版:
		\begin{equation}
			r=\frac{\sum_{i=1}^n\left(X_i-\bar{X}\right)\left(Y_i-\bar{Y}\right)}{\sqrt{\sum_{i=1}^n\left(X_i-\bar{X}\right)^2} \sqrt{\sum_{i=1}^n\left(Y_i-\bar{Y}\right)^2}}
		\end{equation}
		
		相关系数矩阵模版:
		
		\begin{gather*}
			\begin{bmatrix}
				Variable & a & b & c & d & e & f \\
				a & 1 & 1 & 1 & 1 & 1 & 1 \\
				b & 1 & 1 & 1 & 1 & 1 & 1 \\
				c & 1 & 1 & 1 & 1 & 1 & 1 \\
				d & 1 & 1 & 1 & 1 & 1 & 1 \\
				e & 1 & 1 & 1 & 1 & 1 & 1 \\
				f & 1 & 1 & 1 & 1 & 1 & 1 \\
			\end{bmatrix}
		\end{gather*}
	
	
	\subsection{问题二模型的建立与求解}
	
		内容
		
		目标规划函数示例:
		
		\begin{equation}
			\begin{aligned}
				& \max \quad E_k=\frac{S_k \cdot Y_{i, y}-Z_k \cdot X_{i, y}}{\gamma} \\
				& \text { s.t. }\left\{\begin{array}{l}
					Z_k=\frac{Y_{i, y} \cdot \beta_k}{1-\alpha_k} \\
					S_k=X_{2, y}\left(1+V_k\right)\left(1+\beta_k\right) \\
					\beta_k \in\{1, c\} \\
					c>1 \\
					
				\end{array}\right.
			\end{aligned}
		\end{equation}
	
	\subsection{问题三模型的建立与求解}
	
	
	%Part Six
	\section{模型的评价、改进与推广}
	\subsection{模型优点}
	\begin{enumerate}
		\item 
		\item 
		\item 
		\item 
	\end{enumerate}
	
	\subsection{模型缺点}
	\begin{enumerate}
		\item 
		\item 
		\item 
	\end{enumerate}
	
	\subsection{模型的改进}
	\begin{enumerate}
		\item 
		\item 
		\item 
	\end{enumerate}
	
	%Part Seven
	\section{参考文献}
	\vspace{-2em} % 减小上面的间距
	\begin{thebibliography}{9}  
		\bibitem{ref1} 
		\bibitem{ref2} 
		\bibitem{ref3} 
		\bibitem{ref4}   
	\end{thebibliography}
	
	\newpage
	\section*{附录}
	
	附录1:支撑材料的文件列表
	
	
	附录2:初始化代码和数据处理代码
	\begin{lstlisting}[language=python,columns=fullflexible,frame=shadowbox]
		import pandas as pd
		import warnings
		import xlwt
		import numpy as np
		import matplotlib.pyplot as plt
		import pylab
		import seaborn as sns
		from pylab import mpl
		from sklearn.preprocessing import PolynomialFeatures
		from sklearn import linear_model
		from sklearn.model_selection import train_test_split
		from sklearn.ensemble import RandomForestRegressor
		from sklearn import metrics
		import statsmodels.api as sm
		import geatpy as ea
		from scipy import  optimize as opt
		from scipy.optimize import minimize
	\end{lstlisting}
	
\end{document}