%设置页面边距(word标准页面)
\documentclass[a4paper]{article}
\usepackage{geometry}
\geometry{a4paper,left=2.7cm,right=2.7cm,top=2.54cm,bottom=2.54cm}

%导入ctex包
\usepackage[UTF8,heading=true]{ctex}

%设置摘要格式
\usepackage{abstract}
\setlength{\abstitleskip}{0em}
\setlength{\absleftindent}{0pt}
\setlength{\absrightindent}{0pt}
\setlength{\absparsep}{0em}
\renewcommand{\abstractname}{\textbf{\zihao{4}{摘要}}}
\renewcommand{\abstracttextfont}{\zihao{-4}} %设置摘要正文字号

%设置页眉和页脚,只显示页脚居中页码
\usepackage{fancyhdr}
\pagestyle{plain}

%调用数学公式包
\usepackage{amssymb}
\usepackage{amsmath}

%调用浮动包
\usepackage{float}
\usepackage{subfig}
\captionsetup[figure]{labelsep=space} %去除图标题的冒号
\captionsetup[table]{labelsep=space} %去除表格标题的冒号
%设置标题格式
\ctexset {
	%设置一级标题的格式
	section = {
		name={,、},
		number=\chinese{section}, %设置中文版的标题
		aftername=,
	},
	%设置三级标题的格式
	subsubsection = {
		format += \zihao{-4} % 设置三级标题的字号
	}
}


%使得英文字体都为Time NewTown
%\usepackage{times}

%图片包导入
\usepackage{graphicx}
\graphicspath{{figures/}} %图片在当前目录下的figures目录

%参考文献包引入
\usepackage{cite}
\usepackage[numbers,sort&compress]{natbib}

%代码格式
\usepackage{listings}
\usepackage{graphicx}%写入python代码
\usepackage{pythonhighlight}%python代码高亮显示
\lstset{
	%numbers=left, %设置行号位置
	%	numberstyle=\tiny, %设置行号大小
	keywordstyle=\color{blue}, %设置关键字颜色
	commentstyle=\color[cmyk]{1,0,1,0}, %设置注释颜色
	escapeinside=``, %逃逸字符(1左面的键),用于显示中文
	breaklines, %自动折行
	extendedchars=false, %解决代码跨页时,章节标题,页眉等汉字不显示的问题
	xleftmargin=1em,xrightmargin=1em, aboveskip=1em, %设置边距
	tabsize=4, %设置tab空格数
	showspaces=false %不显示空格
}


\renewcommand{\refname}{}

%item包
\usepackage{enumitem}

%表格加粗
\usepackage{booktabs}

%设置表格间距
\usepackage{caption}

%允许表格长跨页
\usepackage{longtable}

%伪代码用到的宏包
\usepackage{algorithmic}
\usepackage{algorithm}

%正文区
\title{论文标题} 
\date{} %不显示日期

%文档
\begin{document}
	\maketitle
	\vspace{-6em} %设置摘要与标题的间距
	\zihao{-4} %设置正文字号
	%摘要部分
	\begin{abstract}
		
		\hspace{0.2em}\textbf{针对问题一},...
		
		\textbf{针对问题二},...
		
		\textbf{针对问题三},...
		
		\textbf{针对问题四},...\\
		%关键词(上文最后一段要用“\\”换行)
		\newline
		\noindent{\textbf{关键词:} \textbf{关键词1}\quad   \textbf{关键词2}\quad \textbf{关键词3} \quad \textbf{关键词4}\quad \textbf{关键词5}} 
	\end{abstract}
	
	\clearpage %换页
	
	%正文部分
	%Part one
	\section{问题背景与重述}
	\subsection{问题背景}
	
	%\begin{figure}[H]
	%	\centering %图片居中
	%	\captionsetup{skip=4pt} % 设置标题与表格的间距为4pt
	%	\includegraphics[width=10cm]{图片文件名} %width设置图片大小
	%	\caption{商超蔬菜示意图\label{商超蔬菜示意图}} %设置图片的标题及引用标签
	%\end{figure}
	
	\subsection{问题重述}
	\begin{enumerate}[itemindent=0.5cm]
		\item ...
		\item ...
		\item ...
		\item ...
	\end{enumerate}
	
	%Part Two
	\section{问题分析}
	\subsection{问题一的分析}
	\subsection{问题二的分析}
	\subsection{问题三的分析}
	\subsection{问题四的分析}
	
	%Part Three
	\section{模型假设}
	%假设的列表
	\begin{enumerate} 
		\item 
		\item 
		\item 
		\item 
	\end{enumerate}
	
	%Part Four
	\section{符号说明}
	%浮动体表格,使用table实现
	\begin{table}[H] %[h]表示在此处添加浮动体,默认为tbf,即页面顶部、底部和空白处添加
		\captionsetup{skip=4pt} % 设置标题与表格的间距为4pt
		\centering
		\setlength{\arrayrulewidth}{2pt} % 设置表格线条宽度为1pt
		\begin{tabular}{cc} %c表示居中,l表示左对齐,r表示右对齐,中间添加“|”表示竖线
			\hline
			\makebox[0.15\textwidth][c]{符号} & \makebox[0.6\textwidth][c]{说明}  \\ 
			\hline
			r & Spearman系数  \\
			&   \\
			&   \\
			&   \\
			\hline
		\end{tabular}
		% \hline是横线,采用\makebox设置列宽
	\end{table}
	
	
	%Part Five
	\section{模型的建立与求解}
	\subsection{问题一模型的建立与求解}
	\subsubsection{三级标题}
	
	计算公式模版:
	\begin{equation}
		r=\frac{\sum_{i=1}^n\left(X_i-\bar{X}\right)\left(Y_i-\bar{Y}\right)}{\sqrt{\sum_{i=1}^n\left(X_i-\bar{X}\right)^2} \sqrt{\sum_{i=1}^n\left(Y_i-\bar{Y}\right)^2}}
	\end{equation}
	
	相关系数矩阵模版:
	
	\begin{gather*}
		\begin{bmatrix}
			Variable & a & b & c & d & e & f \\
			a & 1 & 1 & 1 & 1 & 1 & 1 \\
			b & 1 & 1 & 1 & 1 & 1 & 1 \\
			c & 1 & 1 & 1 & 1 & 1 & 1 \\
			d & 1 & 1 & 1 & 1 & 1 & 1 \\
			e & 1 & 1 & 1 & 1 & 1 & 1 \\
			f & 1 & 1 & 1 & 1 & 1 & 1 \\
		\end{bmatrix}
	\end{gather*}
	
	
	\subsection{问题二模型的建立与求解}
	
	内容
	
	目标规划函数示例:
	
	\begin{equation}
		\begin{aligned}
			& \max \quad E_k=\frac{S_k \cdot Y_{i, y}-Z_k \cdot X_{i, y}}{\gamma} \\
			& \text { s.t. }\left\{\begin{array}{l}
				Z_k=\frac{Y_{i, y} \cdot \beta_k}{1-\alpha_k} \\
				S_k=X_{2, y}\left(1+V_k\right)\left(1+\beta_k\right) \\
				\beta_k \in\{1, c\} \\
				c>1 \\
				
			\end{array}\right.
		\end{aligned}
	\end{equation}
	
	\subsection{问题三模型的建立与求解}
	
	
	%Part Six
	\section{模型的评价、改进与推广}
	\subsection{模型优点}
	\begin{enumerate}
		\item 
		\item 
		\item 
		\item 
	\end{enumerate}
	
	\subsection{模型缺点}
	\begin{enumerate}
		\item 
		\item 
		\item 
	\end{enumerate}
	
	\subsection{模型的改进}
	\begin{enumerate}
		\item 
		\item 
		\item 
	\end{enumerate}
	
	%Part Seven
	\section{参考文献}
	\vspace{-2em} % 减小上面的间距
	\begin{thebibliography}{9}  
		\bibitem{ref1} 
		\bibitem{ref2} 
		\bibitem{ref3} 
		\bibitem{ref4}   
	\end{thebibliography}
	
	\newpage
	\section*{附录}
	
	附录1:支撑材料的文件列表
	
	
	附录2:初始化代码和数据处理代码
	\begin{lstlisting}[language=python,columns=fullflexible,frame=shadowbox]
		import pandas as pd
		import warnings
		import xlwt
		import numpy as np
		import matplotlib.pyplot as plt
		import pylab
		import seaborn as sns
		from pylab import mpl
		from sklearn.preprocessing import PolynomialFeatures
		from sklearn import linear_model
		from sklearn.model_selection import train_test_split
		from sklearn.ensemble import RandomForestRegressor
		from sklearn import metrics
		import statsmodels.api as sm
		import geatpy as ea
		from scipy import  optimize as opt
		from scipy.optimize import minimize
	\end{lstlisting}
	
\end{document}